\documentclass[11pt,]{article}
\usepackage[sc, osf]{mathpazo}
\usepackage{amssymb,amsmath}
\usepackage{ifxetex,ifluatex}
\usepackage{fixltx2e} % provides \textsubscript
\ifnum 0\ifxetex 1\fi\ifluatex 1\fi=0 % if pdftex
  \usepackage[T1]{fontenc}
  \usepackage[utf8]{inputenc}
\else % if luatex or xelatex
  \ifxetex
    \usepackage{mathspec}
  \else
    \usepackage{fontspec}
  \fi
  \defaultfontfeatures{Ligatures=TeX,Scale=MatchLowercase}
\fi
% use upquote if available, for straight quotes in verbatim environments
\IfFileExists{upquote.sty}{\usepackage{upquote}}{}
% use microtype if available
\IfFileExists{microtype.sty}{%
\usepackage{microtype}
\UseMicrotypeSet[protrusion]{basicmath} % disable protrusion for tt fonts
}{}
\usepackage[margin=1in]{geometry}




\setlength{\emergencystretch}{3em}  % prevent overfull lines
\providecommand{\tightlist}{%
  \setlength{\itemsep}{0pt}\setlength{\parskip}{0pt}}
\setcounter{secnumdepth}{0}
% Redefines (sub)paragraphs to behave more like sections
\ifx\paragraph\undefined\else
\let\oldparagraph\paragraph
\renewcommand{\paragraph}[1]{\oldparagraph{#1}\mbox{}}
\fi
\ifx\subparagraph\undefined\else
\let\oldsubparagraph\subparagraph
\renewcommand{\subparagraph}[1]{\oldsubparagraph{#1}\mbox{}}
\fi

% Now begins the stuff that I added.
% ----------------------------------

% Custom section fonts
\usepackage{sectsty}
\sectionfont{\rmfamily\mdseries\large\bf}
\subsectionfont{\rmfamily\mdseries\normalsize\itshape}


% Make lists without bullets
%\renewenvironment{itemize}{
%  \begin{list}{}{
%    \setlength{\leftmargin}{1.5em}
%  }
%}{
%  \end{list}
%}


% Make parskips rather than indent with lists.
\usepackage{parskip}
\usepackage{titlesec}
\titlespacing\section{0pt}{12pt plus 4pt minus 2pt}{4pt plus 2pt minus 2pt}
\titlespacing\subsection{0pt}{12pt plus 4pt minus 2pt}{4pt plus 2pt minus 2pt}

% Use fontawesome. Note: you'll need TeXLive 2015. Update.
\usepackage{fontawesome}

\usepackage[swedish]{babel}  

% Fancyhdr, as I tend to do with these personal documents.
\usepackage{fancyhdr,lastpage}
\pagestyle{fancy}
\renewcommand{\headrulewidth}{0.0pt}
\renewcommand{\footrulewidth}{0.0pt}
\lhead{}
\chead{}
\rhead{}
\lfoot{
\cfoot{\scriptsize  Richard Öhrvall - CV }}
\rfoot{\scriptsize \thepage/{\hypersetup{linkcolor=black}\pageref{LastPage}}}

% Always load hyperref last.
\usepackage{hyperref}
\PassOptionsToPackage{usenames,dvipsnames}{color} % color is loaded by hyperref

\hypersetup{unicode=true,
            pdftitle={Richard Öhrvall:  CV (Curriculum Vitae)},
            pdfauthor={Richard Öhrvall},
            pdfkeywords={RMarkdown, academic CV, template},
            colorlinks=true,
            linkcolor=blue,
            citecolor=Blue,
            urlcolor=blue,
            breaklinks=true, bookmarks=true}
\urlstyle{same}  % don't use monospace font for urls

\begin{document}


\centerline{\huge \bf Richard Öhrvall}

\vspace{2 mm}

\hrule

\vspace{2 mm}

\moveleft.5\hoffset\centerline{Postdok, Centrum för kommunstrategiska studier, Linköpings universitet}
\moveleft.5\hoffset\centerline{Affilierad forskare, Institutet för Näringslivsforskning (IFN)}
\moveleft.5\hoffset\centerline{Vattenledningsvägen 16, 126 33, Hägersten}
\moveleft.5\hoffset\centerline{ \faEnvelopeO \hspace{1 mm} \href{mailto:}{\tt \href{mailto:richard.ohrvall@gmail.com}{\nolinkurl{richard.ohrvall@gmail.com}}} \hspace{1 mm}     \faGlobe \hspace{1 mm} \href{https://richardohrvall.rbind.io/sv}{\tt richardohrvall.rbind.io/sv}    | \emph{Uppdated:} \today   | \emph{Uppdaterat:} \today}

\vspace{2 mm}

\hrule


\hypertarget{utbildning}{%
\section{UTBILDNING}\label{utbildning}}

\emph{Linköpings universitet}, doktorandstudier i statsvetenskap
\hfill 2013--2018

\emph{Stockholms universitet}, statistik, kandidatnivå \hfill 2013

\emph{Uppsala universitet}, fil.mag. statsvetenskap \hfill 2002

Olika kurser vid \emph{Lunds universitet, Örebro universitet, Fudan
University (Kina), Beijing Language and Cultural School (Kina),
University of Michigan (USA)} och olika interna kurser vid
\emph{Statistiska centralbyrån (SCB)}.

\hypertarget{arbetslivserfarenhet}{%
\section{ARBETSLIVSERFARENHET}\label{arbetslivserfarenhet}}

\emph{Centrum för kommunstrategiska studier, Linköpings universitet}
\hfill 2019--2021

\begin{quote}
Postdok
\end{quote}

\emph{Linköpings universitet} \hfill 2013--2018

\begin{quote}
Doktorand i statsvetenskap
\end{quote}

\emph{Institutet för Näringslivsforskning} \hfill 2016--2018

\begin{quote}
Forskare
\end{quote}

\emph{Statistiska centralbyrån (SCB)} \hfill 2001--

\begin{quote}
Ansvarig för valundersökningar, valdeltagandeundersökningar,
undersökningar av folkvalda och förtroendevalda, opinionsundersökningar
och övriga undersökningar inom demokratiområdet. Arbetsuppgifterna
innefattar bland annat undersökningsdesign, frågekonstruktion,
urvalsdesign, estimation, analys, rapportskrivande, etc.
\end{quote}

\begin{quote}
Redaktör för SCB:s tidskrift Välfärd 2005--2006.
\end{quote}

\begin{quote}
Tjänstledig 2006--2007, 2016--2018, 2019--2021.
\end{quote}

\emph{Institutet för Näringslivsforskning} \hfill 2013--2014

\begin{quote}
Anställd på halvtid inom ramen för forskningsprogrammet ``Från
välfärdsstat till välfärdssamhälle''.
\end{quote}

\emph{Linköpings universitet} \hfill 2013--2014

\begin{quote}
Anställd på halvtid under 4 månader inom ramen för forskningsprojektet
''Styrning av, insyn i och kontroll över kommunala bolag''.
\end{quote}

\emph{Linnéuniversitetet} \hfill 2010--2012

\begin{quote}
Anställd på halvtid inom ramen för forskningsprojektet ``Tillit och
korruption i lokalpolitiken''.
\end{quote}

\emph{Linköpings universitet} \hfill 2009

\begin{quote}
Anställd på halvtid under 4 månader vid Centrum för kommunstrategiska
studier (CKS) inom ramen för forskningsprojektet ''Politikens villkor''.
\end{quote}

\emph{Statistiska centralbyrån/Sida} \hfill 2006--2007

\begin{quote}
Arbetade som expert på hushållsundersökningar vid National Institute of
Statistics (NIS) i Kambodja. Kontraktstiden varade från mars 2006 till
och med juli 2007. Formellt sett anställd av SIPU.
\end{quote}

\emph{Uppsala universitet} \hfill 2000--2001

\begin{quote}
Olika anställningar vid Statsvetenskapliga institutionen, Uppsala
universitet: en månads anställning för att skriva en rapport inom
forskningsprojektet ''Demokratins mekanismer'', Timanställd för att
bearbeta data inom forskningsprojektet ``EU:s återverkningar på nordisk
demokrati'', och timanställd för att bearbeta data inom
forskningsprojektet ``Coalition Formation in Swedish Local Government''.
\end{quote}

\hypertarget{utlandserfarenhet}{%
\section{UTLANDSERFARENHET}\label{utlandserfarenhet}}

Långtidsstationerad vid National Institute of Statistics, Kambodja, i
egenskap av expert på hushållsundersökningar, mars 2006--- juli 2007.

Kortare konsultuppdrag för SCB/Sidas räkning i Albanien, Armenien,
Belgien, Bosnien \& Hercegovina, Burundi, Frankrike, Kina, Laos, Mexico
och Serbien, 2002-.

Ansvarig för SCB:s projekt i Armenien och Kina, 2007--2009.

SCB:s representant i det internationella projektet Metagora, vilket
syftade till att ta fram indikatorer för mätning av demokrati, mänskliga
rättigheter och governance, 2003---2008.

\hypertarget{undervisning}{%
\section{UNDERVISNING}\label{undervisning}}

Jag har undervisat i kvantitativa metoder i statsvetenskap på kandidat-
och masternivå vid Linköpings universitet sedan 2014. Detta innefattar
föreläsningar, seminarier och datasalsövningar. Jag har även hållit i
föreläsningar på masternivå om visualisering av data vid Uppsala
universitet och om offentlig förvaltning vid Linköpings universitet.

\hypertarget{publikationer}{%
\section{PUBLIKATIONER}\label{publikationer}}

\hypertarget{artiklar-med-refereegranskning-peer-review}{%
\subsection{\texorpdfstring{\textbf{Artiklar med refereegranskning (peer
review)}}{Artiklar med refereegranskning (peer review)}}\label{artiklar-med-refereegranskning-peer-review}}

``A sticky trait: Social trust among Swedish expatriates in countries
with varying institutional quality'' (2018), \emph{Journal of
Comparative Economics}, 46(4): 1146-1157. {[}Med Andreas Bergh{]}

``Vad mandatperioden gör med fullmäktigeledamoten: Stärks politiskt
självförtroende eller blir man desillusionerad?'' (2017),
\emph{Surveyjournalen}, 3(1). {[}Med Gissur Ó. Erlingsson{]}

More politicians, more corruption: evidence from Swedish municipalities
(2017), \emph{Public Choice}, 172(3-4): 483--500. {[}Med Andreas Bergh
och Günther Fink{]}

''Att bolagisera kommunal verksamhet. Implikationer för granskning,
ansvarsutkrävande och demokrati?'' (2016), \emph{Statsvetenskaplig
tidskrift}, 117(4): 555-585. {[}Med Gissur Ó. Erlingsson, Mattias
Fogelgren, Fredrik Olsson och Anna Thomasson{]}.

``Distrust in Utopia? Public Perceptions of Corruption and Political
Support in Iceland Before and after the Financial Crisis in 2008''
(2016), \emph{Government and Opposition}, 51(4): 553-579. {[}Med Gissur
Ó. Erlingsson och Jonas Linde{]}.

''Not so fair after all? Perceptions of procedural fairness and
satisfaction with democracy in the Nordic welfare states'' (2014),
\emph{International Journal of Public Administration}, 37(2): 106-119.
{[}Med Gissur Ó. Erlingsson och Jonas Linde{]}.

''The Single-Issue Party Thesis and the Sweden Democrats'' (2014),
\emph{Acta Politica}, 49(2): 196-216. {[}Med Gissur Ó. Erlingsson och
Kåre Vernby{]}.

''Does Election Day Weather Affect Voter Turnout? Evidence from Swedish
Elections'' (2014), \emph{Electoral Studies}, 33: 335-342. {[}Med Mikael
Persson och Anders Sundell{]}.

''Voter turnout and political equality: Testing the law of dispersion in
a Swedish natural experiment'' (2013), \emph{Politics}, 33(3): 172-18.
{[}Med Mikael Persson och Maria Solevid{]}.

''Anti-immigrant parties, local presence and electoral success'' (2012),
\emph{Local Government Studies}, 38(6): 817-830. {[}Med Gissur Ó.
Erlingsson och Karl Loxbo{]}.

``Den motvilligt engagerade altruisten: Om partimedlemskap och
partiaktivism'' (2012). \emph{Statsvetenskaplig tidskrift}, 114(2):
185-205. {[}Med Gissur Ó. Erlingsson och Mikael Persson{]}.

``Why Do Councillors Quit Prematurely? On the Democratic Consequences of
Councilors Leaving Their Seats Before the End of Their Terms'' (2011),
\emph{Lex Localis - Journal of Local Self-government}, 9(4), 297-310.
{[}Med Gissur Ó. Erlingsson{]}.

\hypertarget{bocker-bokkapitel-och-rapporter}{%
\subsection{\texorpdfstring{\textbf{Böcker, bokkapitel och
rapporter}}{Böcker, bokkapitel och rapporter}}\label{bocker-bokkapitel-och-rapporter}}

\emph{Growing into Voting: Election Turnout among Young People and Habit
Formation} (2018), avhandling i statsvetenskap, Linköpings universitet.

''Issues On Transparency, Accountability and Control in Hybrid
Organisations: The Case of Enterprises Owned by Local Government''
(2018), i Andrea Bonomi Savignon, Luca Gnan, Alessandro Hinna och Fabio
Monteduro (red.) \emph{Hybridity in the Governance and Delivery of
Public Services}, Emerald Publishing. {[}Med Gissur Ó. Erlingsson och
Anna Thomasson{]}.

\emph{Fullmäktigeledamoten och mandatperioden} (2017), Stockholm:
Sveriges Kommuner och Landsting. {[}Med Gissur Ó. Erlingsson{]}

\emph{A Clean House? Studies of corruption in Sweden} (2016), Lund:
Nordic Academic Press. {[}Med Andreas Bergh, Gissur Ó. Erlingsson och
Mats Sjölin{]}.

``Tål den svenska modellen att jämföras?: Om utlandssvenskars attityder
till välfärdsstaten'' (2016), i Maria Solevid (red.) \emph{Svenska
utlandsröster}. Göteborg: SOM-institutet, sid. 211-232. {[}Med Andreas
Bergh och Henrik Jordahl{]}.

''Voter Turnout'' (2015), kaptiel i Jon Pierre (red.) \emph{Oxford
Handbook on Swedish Politics}. Oxford: Oxford University Press,
sid.229-245.

\emph{Vilka valde att välja? Deltagandet i valen 2014} (2015).
Stockholm: Statistiska centralbyrån.

\emph{Att ta plats i politiken -- om engagemang, aktivism och villkor i
kommunpolitiken} (2015), Stockholm: Sveriges Kommuner och Landsting.
{[}Med Gissur Ó. Erlingsson och Mattias Fogelgren{]}

\emph{Hur styrs och granskas kommunala bolag?} (2014), rapport.
Norrköping: Centrum för kommunstrategiska studier. {[}Med Gissur Ó.
Erlingsson, Mattias Fogelgren, Fredrik Olsson och Anna Thomasson{]}.

``Nationella reformer och lokala initiativ'' (2013), kapitel i Henrik
Jordahl (red.) \emph{Välfärdstjänster i privat regi}. Stockholm: SNS.
{[}Med Henrik Jordahl{]}

\emph{Allmän nytta eller egen vinning? En ESO-rapport om korruption på
svenska} (2013), rapport 2013:2, Expertgruppen för studier i offentlig
ekonomi (ESO), Stockholm: Fritzes {[}Med Andreas Bergh, Gissur Ó.
Erlingsson och Mats Sjölin{]}.

\emph{Svenskt valdeltagande under hundra år} (2012), rapport i serien
Demokratistatistik, Stockholm: Statistiska centralbyrån.

``Omvalet -- en prövning för den politiska jämlikheten'' (2012), kapitel
i Linda Berg och Henrik Oscarsson (red.), \emph{Omstritt omval},
Göteborg: Göteborgs universitet. {[}Med Maria Solevid och Mikael
Persson{]}.

\emph{Valdeltagande vid omvalen 2011} (2012), Stockholm: Statistiska
centralbyrån.

``Regional och lokal tillväxtpolitik. Vad kan och bör offentliga aktörer
göra?'' (2011), Working paper 2011:28, Myndigheten för tillväxtpolitiska
utvärderingar och analyser. {[}Med Gissur Ó. Erlingsson och Jerker
Moodysson{]}

\emph{Politikens villkor. Om engagemang och avhopp i kommunpolitiken}
(2010), rapport 2010:4, Linköping: Centrum för kommunstrategiska
studier, Linköping universitet. {[}Med Gissur Ó. Erlingsson{]}

\emph{Valdeltagande på valdistriktsnivå} (2009), rapport, Stockholm:
Sveriges Kommuner och Landsting.

\emph{Valdeltagande bland förstagångsväljare} (2009), rapport,
Stockholm: Sveriges Kommuner och Landsting.

\emph{Förtida röstning i Sverige} (2008), Göteborg: Göteborgs
universitet. {[}Med Stefan Dahlberg och Henrik Oscarsson{]}

''Demokrati'', kapitel i Karin E. Lundström m.fl., \emph{Integration --
en beskrivning av läget i Sverige} (2008). Stockholm: Statistiska
centralbyrån.

\emph{Förtroendevalda i kommuner och landsting 2007. En rapport om
politikerantal och representativitet} (2008), Stockholm: Statistiska
centralbyrån. {[}Med Jessica Persson{]}

''Invandrade och valdeltagande'' (2006), kapitel i Hanna Bäck och Mikael
Gilljam (red.), Valets mekanismer, Malmö: Liber.

''Väljare och valda -- några avtryck i valstatistiken'' (2005), kapitel
i \emph{Hundre års ensomhet? Norge og Sverige 1905 -- 2005}. Oslo:
Statistisk sentralbyrå. {[}Med Mikaela Järnbert och Staffan Sollander{]}

\emph{Hel- och deltidsarvoderade förtroendevalda} (2004), rapport,
Stockholm: Statistiska centralbyrån.

\emph{Det nya seklets förtroendevalda: Om politikerantal och
representativitet i kommuner och landsting 2003} (2004), Stockholm:
Justitiedepartementet, Kommunförbundet och Landstingsförbundet. {[}Med
Hanna Bäck{]}

\emph{Ja och nej till euron} (2004), Stockholm: Statistiska
centralbyrån. {[}Med Staffan Sollander{]}

''Det svenska valdeltagandet'' (2003), kapitel i Joachim Vogel (red.),
\emph{Välfärd och ofärd på 90-talet}, Stockholm: Statistiska
centralbyrån. {[}Med Mikaela Järnbert{]}

\hypertarget{debattartiklar-i-urval}{%
\subsection{\texorpdfstring{\textbf{Debattartiklar (i
urval)}}{Debattartiklar (i urval)}}\label{debattartiklar-i-urval}}

``Vässa styrningen av kommunala bolag'' {[}Med Gissur Ó. Erlingsson,
Mattias Fogelgren, Fredrik Olsson och Anna Thomasson{]}. \emph{Dagens
Samhälle}, 27 november 2014.

''Granskaren måste komma utifrån'' {[}Med Andreas Bergh, Gissur Ó.
Erlingsson och Mats Sjölin{]}, \emph{Balans}, 3 juni 2013.

''Svenskar misstror offentliga tjänstemäns ärlighet'' {[}Med Andreas
Bergh, Gissur Ó. Erlingsson och Mats Sjölin{]}, \emph{DN Debatt}, 9
april 2013.

\hypertarget{working-papers-och-ovriga-artiklar-i-urval}{%
\subsection{\texorpdfstring{\textbf{Working papers och övriga artiklar
(i
urval)}}{Working papers och övriga artiklar (i urval)}}\label{working-papers-och-ovriga-artiklar-i-urval}}

``Voices from the far right: a text analysis of Swedish parliamentary
debates'', working paper, SocArxiv, (under review) {[}Med Måns
Magnusson, Katarina Barrling och David Mimno{]}

``Privata aktörer inom arbetsförmedling och rehabilitering'', \emph{SNS
Analys}, nummer 11, Stockholm: SNS, 2013.

''Kommunerna brister i hanteringen av anställdas bisysslor'' {[}Med
Gissur Ó. Erlingsson{]}, artikel i \emph{Ekonomisk Debatt}, 2013:3.

''Kvinnorna förbi i valdeltagande'', artikel i SCB:s tidskrift
\emph{Välfärd}, 2013:1.

``Fler äldre män bland valda än bland väljare'', artikel i SCB:s
tidskrift \emph{Välfärd}, 2012:1.

``Januaribarn mer framgångsrika'' {[}Med Lotta Persson{]}, artikel i
SCB:s tidskrift \emph{Välfärd}, 2011:3.

``Partiernas sympatisörer'' {[}Med Mikaela Järnbert{]}, artikel i SCB:s
tidskrift \emph{Välfärd}, 2010:2.

``Förtida avhopp'' {[}Med Gissur Ó. Erlingsson{]}, artikel i SCB:s
tidskrift \emph{Välfärd}, 2010:2.

``Få röstar i val till Europaparlamentet'' {[}Med Jessica Persson{]},
artikel i SCB:s tidskrift \emph{Välfärd}, 2009:2.

``80-talistkvinnor långt till vänster'', artikel i SCB:s tidskrift
\emph{Välfärd}, 2009:1.

``Värdet av en röst'', artikel i SCB:s tidskrift \emph{Välfärd}, 2008:3.

``Tydliga könsmönster i politiken'' {[}Med Jessica Persson{]}, artikel i
SCB:s tidskrift \emph{Välfärd}, 2008:2.

``Större tillgänglighet gav fler förtidsröster'' {[}Med Stefan Dahlberg
och Henrik Oscarsson{]}, artikel i SCB:s tidskrift \emph{Välfärd},
2007:4.

``Avhopp i politiken vanligare bland kvinnor'', artikel i SCB:s
tidskrift \emph{Välfärd}, 2007:3.

``Allt färre 30- och 40-åringar i politiken'', artikel i SCB:s tidskrift
\emph{Välfärd}, 2006:2.

``Som Ljungby röstar, röstar inte Sverige'', artikel i SCB:s tidskrift
\emph{Välfärd}, 2006:1.

``Den glömda generationen -- den röda generationen'' {[}Med Staffan
Sollander{]}, artikel i SCB:s tidskrift \emph{Välfärd}, 2005:4.

``Svenskt valdeltagande bland de lägsta i EU'', artikel i SCB:s
tidskrift \emph{Välfärd}, 2005:2.

\hypertarget{stipendier-och-utmarkelser}{%
\section{STIPENDIER OCH UTMÄRKELSER}\label{stipendier-och-utmarkelser}}

Pris för bästa artikel i Statsvetenskaplig tidskrift år 2012 med
artikeln ''Den motvilligt engagerade altruisten: Om partimedlemskap och
partiaktivism'', skriven tillsammans med Gissur Ó. Erlingsson och Mikael
Persson.

Uppsala kommuns uppsatsstipendium, 2001.

Uppsala universitet, stipendium för studier i Kina, 1999.

\hypertarget{ovrigt}{%
\section{ÖVRIGT}\label{ovrigt}}

Jag har fått artiklar antagna för presentation vid konferenser anordnade
av bland andra European Group for Public Administration (EGPA),
Political Studies Association (PSA), Midwestern Political Science
Association (MPSA) och American Association for Public Opinion Research
(AAPOR). Vidare har jag expertgranskat artiklar för olika vetenskapliga
tidskrifter, såsom European Journal of Political Research och Journal of
Comparative Policy Analysis. Jag har föreläst vid olika seminarier,
konferenser och kurser, bland annat var jag under hösten 2012
gästföreläsare vid Uppsala universitets mastersutbildning i
statsvetenskap och 2013 vid Örebro universitets sommarkurs Analysis of
Survey Data. Sommaren 2013 höll jag en av de inledande föreläsningarna
vid det nordiska statistikermötet i Bergen. Under år 2014 hade jag hos
Svenska Dagbladet bloggen 312, vilken behandlade val och statistik. Jag
är en skribenterna bakom den statsvetenskapliga bloggen Politologerna.
Jag är affilierad till Institutet för analytisk sociologi (IAS), vid
Linköpings universitet.

Jag har erfarenhet av arbeta med statistikprogrammen SAS, SPSS, Stata
och R, samt datorprogrammet Python.

\hypertarget{referenser}{%
\section{REFERENSER}\label{referenser}}

Ges vid anmodan.

\end{document}
