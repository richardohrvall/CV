\documentclass[11pt,]{article}
\usepackage[sc, osf]{mathpazo}
\usepackage{amssymb,amsmath}
\usepackage{ifxetex,ifluatex}
\usepackage{fixltx2e} % provides \textsubscript
\ifnum 0\ifxetex 1\fi\ifluatex 1\fi=0 % if pdftex
  \usepackage[T1]{fontenc}
  \usepackage[utf8]{inputenc}
\else % if luatex or xelatex
  \ifxetex
    \usepackage{mathspec}
  \else
    \usepackage{fontspec}
  \fi
  \defaultfontfeatures{Ligatures=TeX,Scale=MatchLowercase}
\fi
% use upquote if available, for straight quotes in verbatim environments
\IfFileExists{upquote.sty}{\usepackage{upquote}}{}
% use microtype if available
\IfFileExists{microtype.sty}{%
\usepackage{microtype}
\UseMicrotypeSet[protrusion]{basicmath} % disable protrusion for tt fonts
}{}
\usepackage[margin=1in]{geometry}




\setlength{\emergencystretch}{3em}  % prevent overfull lines
\providecommand{\tightlist}{%
  \setlength{\itemsep}{0pt}\setlength{\parskip}{0pt}}
\setcounter{secnumdepth}{0}
% Redefines (sub)paragraphs to behave more like sections
\ifx\paragraph\undefined\else
\let\oldparagraph\paragraph
\renewcommand{\paragraph}[1]{\oldparagraph{#1}\mbox{}}
\fi
\ifx\subparagraph\undefined\else
\let\oldsubparagraph\subparagraph
\renewcommand{\subparagraph}[1]{\oldsubparagraph{#1}\mbox{}}
\fi

% Now begins the stuff that I added.
% ----------------------------------

% Custom section fonts
\usepackage{sectsty}
\sectionfont{\rmfamily\mdseries\large\bf}
\subsectionfont{\rmfamily\mdseries\normalsize\itshape}


% Make lists without bullets
%\renewenvironment{itemize}{
%  \begin{list}{}{
%    \setlength{\leftmargin}{1.5em}
%  }
%}{
%  \end{list}
%}


% Make parskips rather than indent with lists.
\usepackage{parskip}
\usepackage{titlesec}
\titlespacing\section{0pt}{12pt plus 4pt minus 2pt}{4pt plus 2pt minus 2pt}
\titlespacing\subsection{0pt}{12pt plus 4pt minus 2pt}{4pt plus 2pt minus 2pt}

% Use fontawesome. Note: you'll need TeXLive 2015. Update.
\usepackage{fontawesome}

  

% Fancyhdr, as I tend to do with these personal documents.
\usepackage{fancyhdr,lastpage}
\pagestyle{fancy}
\renewcommand{\headrulewidth}{0.0pt}
\renewcommand{\footrulewidth}{0.0pt}
\lhead{}
\chead{}
\rhead{}
\lfoot{
\cfoot{\scriptsize  Richard Öhrvall - CV }}
\rfoot{\scriptsize \thepage/{\hypersetup{linkcolor=black}\pageref{LastPage}}}

% Always load hyperref last.
\usepackage{hyperref}
\PassOptionsToPackage{usenames,dvipsnames}{color} % color is loaded by hyperref

\hypersetup{unicode=true,
            pdftitle={Richard Öhrvall:  CV (Curriculum Vitae)},
            pdfauthor={Richard Öhrvall},
            pdfkeywords={RMarkdown, academic CV, template},
            colorlinks=true,
            linkcolor=blue,
            citecolor=Blue,
            urlcolor=blue,
            breaklinks=true, bookmarks=true}
\urlstyle{same}  % don't use monospace font for urls

\begin{document}


\centerline{\huge \bf Richard Öhrvall}

\vspace{2 mm}

\hrule

\vspace{2 mm}

\moveleft.5\hoffset\centerline{Researcher, Research Institute of Industrial Economics (IFN)}
\moveleft.5\hoffset\centerline{Senior advisor, Statistics Sweden}
\moveleft.5\hoffset\centerline{Vattenledningsvägen 16, SE-126 33, Hägersten, Sweden}
\moveleft.5\hoffset\centerline{ \faEnvelopeO \hspace{1 mm} \href{mailto:}{\tt \href{mailto:richard.ohrvall@gmail.com}{\nolinkurl{richard.ohrvall@gmail.com}}} \hspace{1 mm}     \faGlobe \hspace{1 mm} \href{https://richardohrvall.rbind.io/en}{\tt richardohrvall.rbind.io/en}    | \emph{Uppdated:} \today  }

\vspace{2 mm}

\hrule


\hypertarget{education}{%
\section{EDUCATION}\label{education}}

\emph{Linköping University}, Ph.D.~in political science \hfill 2018

\emph{Stockholm University}, statistics, bachelor's level \hfill 2017

\emph{Uppsala universitet}, M.A.~in political science \hfill 2002

Different courses at \emph{Lund University, Örebro University, Fudan
University (China), Beijing Language and Cultural School (China),
University of Michigan (USA)} and internal courses at \emph{Statistics
Sweden}.

\hypertarget{employment}{%
\section{EMPLOYMENT}\label{employment}}

\emph{Research Institute of Industrial Economics} \hfill 2016--

\begin{quote}
Researcher
\end{quote}

\emph{Statistics Sweden} \hfill 2001--

\begin{quote}
Senior advisor, responsible for the Swedish National Election Studies
(SNES), the Swedish Electoral Participation Survey, the Study of
Nominated and Elected Candidates, and The Survey of Elected
Representatives in Municipalities and County Councils,
\end{quote}

\begin{quote}
Editor of Statistics Sweden's journal \emph{Välfärd} {[}Welfare{]}
2005--2006.
\end{quote}

\begin{quote}
On leave 2006--2007, 2016--2018
\end{quote}

\emph{Linköping Universitet} \hfill 2013--2018

\begin{quote}
PhD student in political science
\end{quote}

\emph{Research Institute of Industrial Economics} \hfill 2013--2014

\begin{quote}
Researcher, the research project ''From Welfare State to Welfare
Society''.
\end{quote}

\emph{Linköping university} \hfill 2013--2014

\begin{quote}
Researcher, research project on municipal corporations.
\end{quote}

\emph{Linnaeus University} \hfill 2010--2012

\begin{quote}
Researcher, research project on trust and corruption in local politics.
\end{quote}

\emph{Linköping University} \hfill 2009

\begin{quote}
Researcher, research project on local politicians.
\end{quote}

\emph{Statistics Sweden/Sida} \hfill 2006--2007

\begin{quote}
Expert on household surveys at National Institute of Statistics (NIS) in
Cambodia, as part of a project founded by Sida.
\end{quote}

\emph{Uppsala University} \hfill 2000--2001

\begin{quote}
Research assistant at Department of Government, Uppsala University, in
the research projects ''Mechanisms of Democracy'', ``Effects of EU On
Nordic Democracies'' , and ``Coalition Formation in Swedish Local
Government''.
\end{quote}

\hypertarget{international-experience}{%
\section{INTERNATIONAL EXPERIENCE}\label{international-experience}}

Long-term advisor, at National Institute of Statistics (NIS), Cambodia,
serving as an expert on household surveys, 2006---2007.

Consultancy missions on behalf of Statistics Sweden and the Swedish
International Cooperation Agency (Sida) in countries such as Albania,
Armenia, Belgium, Bosnia \& Herzegovina, Burundi, France, China, Laos,
Mexico and Serbia, 2002---2015.

Project manager for Statistics Sweden's projects in Armenia and China,
2007---2009.

Statistics Sweden's representative in the international project
Metagora, under OECD and with the purpose of creating indicators to
measure democracy, human rights and governance, 2003---2008.

\hypertarget{teaching}{%
\section{TEACHING}\label{teaching}}

I have been teaching quantitative methods at B.A and M.A level at
Linköping University since 2014, including lectures, seminars and
computer labs. I have also held lectures (M.A.~level) on visualization
of data at Uppsala University and on quality of government at Linköping
University.

\hypertarget{publications}{%
\section{PUBLICATIONS}\label{publications}}

I have written 11 articles published in scientic journals with peer
review, and also several reports, books and book chapters. For a list of
publications, see
\href{https://richardohrvall.rbind.io/en/publication/}{my website}.

\end{document}
